\documentclass[a4paper]{article}
\usepackage[brazilian]{babel}
\usepackage{subfig}
\usepackage{booktabs}
\usepackage{graphicx}
%-----------------------------------------------------------------
\title{Projeto 2: Simula\c{c}\~ao de modelo presa-predador com acidente com poluente.}
\author{122830 Alcides Goldoni Junior\\
  \small MS 680 - Modelos matem\'{a}ticos aplicados a Biologia\\
  %\small Unicamp 
}%Fechando Title
\begin{document}
\maketitle
%Seções
%-----------------------------------------------------------------
\section{Introdu\c{c}\~{a}o}
\\
Nesse projeto irei simular a preda\c{c}\~ao de tr\^es esp\'ecies por um predador que se alimenta exclusivamente dessas esp\'ecies.
\\
Em um determinado instante no tempo, ser\'a introduzido no meio, um acidente com poluente, onde ele \'e despejado de uma s\'o vez e decai ao longo do tempo. Esse poluente afeta apenas as esp\'ecies predadas.
\\
Vou analisar o compotamento das esp\'ecies antes e depois do acidente, verificando se existe estabilidade na conviv\^encia ou se elas chegar\~ao a extin\c{c}\~ao
\section{Modelagem}
Tr\^es esp\'ecies vivem em um lago suficientemente grande. Essas esp\'ecies competem pelos recursos do meio ambiente mas n\~ao se predam. Nesse mesmo lago, existe uma quarta esp\'ecie que preda as outras tr\^es.
\\
Um acidente ocorre e despeja um certa quantidade de poluente nesse lago, afetando negativamente as esp\'ecies que s\~ao predadas mas n\~ao afetando o predador. O poluente sofre um decaimento ao longo do tempo de forma muito lenta, afetando outras gera\c{c}\~oes da popula\c{c}\~ao de presas.
\\
Partindo das premissas acima, vou modelar a equa\c{c}\~ao que descreve a situa\c\{c}\~ao utilizando do modelo de crescimento populacional descrito por Verhulst e um decaimento constante para o poluente.

\begin{equation}
\left\{\begin{array}{l}
\frac{\delta A}{\delta t} =  \lambda_A (A + B + C + D)(2 - \frac{A + B + C +D}{K}) -\alpha_1AD - \alpha_2PA  \\
\\
\frac{\delta B}{\delta t} =  \lambda_B (A + B + C + D)(1 - \frac{A + B + C +D}{K}) -\beta_1BD - \beta_2BA  \\
\\
\frac{\delta C}{\delta t} =  \lambda_C (A + B + C + D)(1 - \frac{A + B + C +D}{K}) -\gamma_1CD - \gamma_2CA  \\
\\
%\frac{\delta D}{\delta t} =  \lambda_D (A + B + C + D)(1 - \frac{A + B + C +D}{K}) + \alpha_1AD + \beta_1 BD + \gamma_1CD  \\
\frac{\delta D}{\delta t} =  \lambda_D (A + B + C + D)(1 - \frac{A + B + C +D}{K}) + \mu_1AD + \mu_2 BD + \mu_3CD  \\
\\
\frac{\delta P}{\delta t} = - \varepsilon P \\
\end{array}
\end{equation}
\\
Onde:
\\
$A, B$ e $C$ s\~ao as esp\'ecies predadas;
\\
$D$ \'e a esp\'ecie predadora;
\\
$P$ \'e o poluente;
\\
$\lambda$ \'e a taxa de crescimento de cada popula\c{c}\~ao;
\\
$\alpha_1$ \'e a taxa de crescimento da popula\c{c}\~ao $A$;
\\
$\alpha_2$ \'e a taxa de decrescimento da popula\c{c}\~ao $A$;
\\
$\beta_1$ \'e a taxa de crescimento da popula\c{c}\~ao $B$;
\\
$\beta_2$ \'e a taxa de decrescimento da popula\c{c}\~ao $B$;
\\
$\gamma_1$ \'e a taxa de crescimento da popula\c{c}\~ao $C$;
\\
$\gamma_2$ \'e a taxa de decrescimento da popula\c{c}\~ao $C$;
\\
$\mu$ \'e a taxa de crescimento da popula\c{c}\~ao $D$ em rela\c{c}\~ao a cada presa.
\\
\\
\'E esperando que quando as presas comecem a diminuir devido a polui\c{c}\~ao, o predador, em um primeiro momento, continue crescendo, mas, em seguida comece a diminuir devido a falta de alimentos.
\\
J\'a as presas, \'e esperado que elas cres\c{c}am at\'e que o acidente aconte\c{c}a, ap\'os isso, \'e esperado que elas diminuam bruscamente at\'e que estabilizem.
\\
N\~ao \'e esperado que nenhuma das popula\c{c}\~oes cheguem a extin\c{c}\~ao, mas dependendo do decaimento do poluente, isso pode ocorrer.\\

\section{Simula\c{c}\~oes}
Para a simula\c{c}\~ao, foi utilizado a linguagem de programa\c{c}\~ao Python e a biblioteca Matplotlib para gerar os gr\'aficos. A simula\c{c}\~ao ocorreu em um notebook Dell Latide E6510 com processador Intel(R) Core(TM) i7 CPU Q 840 @ 1.87GHz, 1TB de HD e 8GB de mem\'oria RAM.
\\
Os valores iniciais da popula\c{c}\~ao das presas s\~ao significativamente maior do que o valor inicial do predador para simular que as presas s\~ao peixes bem menores que o predador e, dessa forma, tem uma popula\c{c}\~ao maior no lago.
\\
J\'a o valor inicial da polui\c{c}\~ao \'e zero, pois nessa simula\c{c}\~ao o acidente ocorre semanas depois do inicio da simula\c{c}\~ao.
\\
Da mesma forma que os valores iniciais de cada popula\c{c}\~ao, a taxa de crescimento das presas \'e significativamente maior que a taxa de crescimento do predador.
\\
O valor do decaimento da polui\c{c}\~ao foi escolhido de forma que a polui\c{c}\~ao permancesse no lago por tempo suficientemente grande para que atingisse v\'arias gera\c{c}\~oes, tanto das presas quanto do predador.
\\
A capacidade de suporte foi escolhida de forma aleat\'oria. Sendo apenas um n\'umero suficientemente maior que a soma dos valores iniciais.
\\

\section{Conclus\~ao}
% Bibliografia
%-----------------------------------------------------------------
\begin{thebibliography}{99}
\bibitem {Cd94} http://www.ibge.gov.br/home/geociencias/cartografia/default\_territ\_area.shtm
\end{thebibliography}

\end{document}
