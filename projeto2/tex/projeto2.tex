\documentclass[a4paper]{article}
\usepackage[brazilian]{babel}
\usepackage{subfig}
\usepackage{booktabs}
\usepackage{graphicx}
%-----------------------------------------------------------------
\title{Projeto 2: Simula\c{c}\~ao de modelo presa-predador com acidente com poluente.}
\author{122830 Alcides Goldoni Junior\\
  \small MS 680 - Modelos matem\'{a}ticos aplicados a Biologia\\
  %\small Unicamp 
}%Fechando Title
\begin{document}
\maketitle
\abstract{}
%Seções
%-----------------------------------------------------------------
\section{Introdu\c{c}\~{a}o}
\\
Nesse projeto irei sumular a preda\c{c}\~ao de tr\^es esp\'ecies por um predador que se alimenta exclusivamente dessas esp\'ecies.
\\
Em um determinado instante no tempo, ser\'a introduzido no meio, um acidente com poluente, onde ele \'e despejado de uma s\'o vez e decai ao longo do tempo. Esse poluente afeta apenas as esp\'ecies predadas.
\\
Vou analisar o compotamento das esp\'ecies antes e depois do acidente, verificando se existe estabilidade na conviv\^encia ou se elas chegar\~ao a extin\c{c}\~ao
\section{Modelagem}
Tr\^es esp\'ecies vivem em um lago suficientemente grande. Essas esp\'ecies competem pelos recursos do meio ambiente mas n\~ao se predam. Nesse mesmo lago, existe uma quarta esp\'ecie que preda as outras tr\^es.
\\
Um acidente ocorre e despeja um certa quantidade de poluente nesse lago, afetando negativamente as esp\'ecies que s\~ao predadas mas n\~ao afetando o predador. O poluente sofre um decaimento ao longo do tempo de forma muito lenta, afetando outras gera\c{c}\~oes da popula\c{c}\~ao de presas.
\\
Partindo das premissas acima, vou modelar a equa\c{c}\~ao que descreve a situa\c\{c}\~ao utilizando do modelo de crescimento populacional descrito por Verhulst e um decaimento constante para o poluente.

\begin{equation}
\left\{\begin{array}{l}
\frac{\delta A}{\delta t} =  \lambda_A (A + B + C + D)(2 - \frac{A + B + C +D}{K}) -\alpha_1AD - \alpha_2PA  \\
\\
\frac{\delta B}{\delta t} =  \lambda_B (A + B + C + D)(1 - \frac{A + B + C +D}{K}) -\beta_1BD - \beta_2BA  \\
\\
\frac{\delta C}{\delta t} =  \lambda_C (A + B + C + D)(1 - \frac{A + B + C +D}{K}) -\gamma_1CD - \gamma_2CA  \\
\\
\frac{\delta D}{\delta t} =  \lambda_D (A + B + C + D)(1 - \frac{A + B + C +D}{K}) + \alpha_1AD + \beta_1 BD + \gamma_1CD  \\
\\
\frac{\delta P}{\delta t} = - \varepsilon P \\
\end{array}
\end{equation}
\\
Onde:
\\
$A, B$ e $C$ s\~ao as esp\'ecies predadas;
\\
$D$ \'e a esp\'ecie predadora;
\\
$P$ \'e o poluente;
\\
$\lambda$ \'e a taxa de crescimento de cada popula\c{c}\~ao;
\\
$\alpha_1$ \'e a taxa de crescimento da popula\c{c}\~ao $A$;
\\
$\alpha_2$ \'e a taxa de decrescimento da popula\c{c}\~ao $A$;
\\
$\beta_1$ \'e a taxa de crescimento da popula\c{c}\~ao $B$;
\\
$\beta_2$ \'e a taxa de decrescimento da popula\c{c}\~ao $B$;
\\
$\gamma_1$ \'e a taxa de crescimento da popula\c{c}\~ao $C$;
\\
$\gamma_2$ \'e a taxa de decrescimento da popula\c{c}\~ao $C$;
\section{Simula\c{c}\~oes}
\section{Conclus\~ao}
% Bibliografia
%-----------------------------------------------------------------
\begin{thebibliography}{99}
\bibitem {Cd94} http://www.ibge.gov.br/home/geociencias/cartografia/default\_territ\_area.shtm
\end{thebibliography}

\end{document}
