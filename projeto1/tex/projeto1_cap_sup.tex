\documentclass[a4paper]{article}
\usepackage[brazilian]{babel}
\usepackage{subfig}
\usepackage{booktabs}
\usepackage{graphicx}
%-----------------------------------------------------------------
\title{Projeto 1: C\'{a}lculo da capacidade de suporte {K} do Brasil }
\author{122830 Alcides Goldoni Junior\\
  \small MS 680 - Modelos matem\'{a}ticos aplicados a Biologia\\
  %\small Unicamp 
}%Fechando Title
\begin{document}
\maketitle
%Abstract
%-----------------------------------------------------------------
Essas s\~ao as notas de aula de Alcides Goldoni Junior, baseado no livro de Fundamentos de F\'isica, 8ª edi\c{c}\~ao de David Halliday, Robert Resnick e Jearl Walker.
%-----------------------------------------------------------------
\section{Cargas El\'etricas}
A carga el\'etrica \'e uma propriedade intr\'iseca das part\'iculas fundamentas de que que \'e feita a mat\'eria, em outras palavras, \'e um apropriedade associada a pr\'opria  exist\^encia dessa part\'icula.
\\
O objeto esta eletricamente \bold neutro quando a quantidade de cargas positivas \'e igual a quantidade de cargas negativas.
\\
Quando a quantidade de cargas total n\~ao \'e zero (ou seja, n\~ao \'e eletricamente neutro), dizemos que o objeto esta eletricamente carregado.
\\
Objetos carregados  exercem for\c{c}as uns osbre os outros.
\\
Cargas de mesmo sinal repelem-se; cargas de sinal diferentes se atraem.
\\
\subsection{Condutores e isolantes}
Os \bold condutores s\~ao materias nos quais as cargas el\'etricas se movem com facilidade.
\\
Os \bold N\~ao-\bold condutores ou \bold isolantes s\~ao materias nos quais as cargas n\ão podem se mover.
\\
Os \bold Semi-\bold condutores s\~ao materias com propriedades el\'etricas intermedi\'arias.
\\



% Bibliografia
%-----------------------------------------------------------------
\begin{thebibliography}{99}
\bibitem{Cd94}  
\end{thebibliography}

\end{document}
