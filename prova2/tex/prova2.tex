\documentclass[a4paper]{article}
\usepackage[brazilian]{babel}
\usepackage{subfig}
\usepackage{booktabs}
\usepackage{graphicx}
%-----------------------------------------------------------------
\title{MS 680 - Modelos matem\'{a}ticos aplicados a Biologia}
\author{122830 Alcides Goldoni Junior\\
  \Small Primeira prova \\
}%Fechando Title
\begin{document}
\maketitle
%----------------------------------------------------------------
%Seções
%----------------------------------------------------------------
\begin{enumerate}
%Questao 1-------------------------------------------------------
\item
\\
%Questao 2-------------------------------------------------------
\item
\\
%Questao 3-------------------------------------------------------
\item
De tudo o que vimos na disciplina, o fato que me deixou mais surpreso foi um exemplo, dado apenas como hist\'oria vivida pelo prof. Joni, da constru\c{c}\~ao de uma represa, onde foram retirados v\'arios animais da \'area que seria alagada e recolocados em outro lugar.
\\
A atitude me parecia coerente, at\'e come\c{c}ar a estudar a Capacidade de Suporte, ap\'os isso, pude entender que ao inv\'es dos animais daquela regi\~ao morrerem devido ao alagamento, por terem sido removidos a um lugar que n\~ao seria alagados, eles acabaram morrendo de fome, afinal, o meio n\~ao estava preparado para manter todos aqueles animais.
\\
Com um estudo r\'apido, que qualquer aluno que tenha cursado essa disciplina poderia fazer, poderia ser mostrado que remover os animais de uma regi\~ao e levar pra outra deveria respeitar um conceito simples, por\'em muito importante, que \'e a capacidade de suporte do meio. 
\\
%Questao 4-------------------------------------------------------
\item
Um outro problema sobre \'e
\\
%Questao 5-------------------------------------------------------
\item
De acordo com a ementa da disciplina, abordamos de forma muito boa os problemas de din\^amica populacional por meio das equa\c{c}\~oes recursivas e equa\c{c}\~oes dferenciais, tendo muitos exemplos em aula e nos dois projetos. O assunto que gostaria de ter estudado com mais enfâse são os processos fisiol\'ogicos, principalmente aqueles que envolvem doen\c{c}as tais como a dengue nos seres humanos, a febre aftosa em gado, a podrid\~ao (vermelha, fusarium ou abacaxi) da cana-de-a\c{c}ucar, entre outros. A grande maioria dos exemplos dado em sala, foram relacionados a dinâmica populacional com ou sem presa-predador e decaimento de poluentes e como isso afeta popula\c{c}\~oes. Mais exemplos com doen\c{c}as, tamb\'em daria mais dinamismo a aula.
\\
%Questao 6-------------------------------------------------------
\item
Um do modelo que poderia ter sido exposto durante as aulas, que foi apenas comentado, \'e o modelo de propaga\c{c}\~ao de doen\c{c}\~as, como o que foi pedido na quest\~ao 2. Vou modelar apenas o modelo SIR para que este seja diferente da quest\~ao acima. Portanto, um doen\c{c}a afeta pessoas infectando elas por um certo per\'iodo de tempo. Pessoas uma vez infectadas tornam-se resistentes ou morrem. A doen\c{c}a \'e de curto per\'iodo de tempo, portanto, n\~ao afeta outras gera\c{c}\~oes.
\\
%----------------------------------------------------------------
\end{enumerate}
\end{document}
