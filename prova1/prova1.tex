\documentclass[a4paper]{article}
\usepackage[brazilian]{babel}
\usepackage{subfig}
\usepackage{booktabs}
\usepackage{graphicx}
%-----------------------------------------------------------------
\title{MS 680 - Modelos matem\'{a}ticos aplicados a Biologia}
\author{122830 Alcides Goldoni Junior\\
  \small Primeira prova \\
}%Fechando Title
\begin{document}
\maketitle
%-----------------------------------------------------------------
%Seções
%-----------------------------------------------------------------
\begin{enumerate}

%Questao 1-----------------------------------------------------------------
\item
Para a situa\c{c}\~ao do conv\'ivio de tr\^es esp\'ecies: gavi\~ao, cobra e roedores, onde os gavi\~oes predam as cobras e os roedores e as cobras predam os roedores, vamos modelar um tripla din\^amica populacional usando o seguinte sistema de equa\c{c}\~oes:
%aqui começa o modelo matematica da dinâmica triplice
\begin{equation}
\end{equation}


%Questao 2-----------------------------------------------------------------
\item
%Questao 3-----------------------------------------------------------------
\item
\\
O que achei mais \'util nessaa disciplina \'e que ela \'e a \'unica que estou fazendo nesse semestre que tem, de fato, uma aplica\c{c}\~ao no mundo real, deixando de tratar problemas bobos em exerc\'icios propostos no livro texto e tratando problemas que podem realmente acontecer no mundo real. Essa abordagem de problemas reais \'e o que acho mais importante, n\~ao s\'o nessa disciplina, mas tamb\'em em qualquer outra da matem\'atica aplicada e muitas vezes n\~ao \'e o que acontece.
\\
J\'a o que achei menos importante foi
%Questao 4-----------------------------------------------------------------
\item
Minha maior dificuldade em rela\c{c}\~ao ao primeiro projeto foi pensar em quais hip\'oteses utilizar e como inserir essas hip\'oteses, que s\~ao fen\^omenos reais, nas equa\c{c}\~oes, dessa forma, achei melhor fazer o projeto com as hip\'oteses que foram utlizadas em sala de aula, pois j\'a estava familiarizado com a "cara" que a equa\c{c}\~ao teria.
%Questao 5-----------------------------------------------------------------
\item
Por se tratar de uma disciplina que trabalha com problemas reais, o ponto que vejo como o mais fraco \'e a falta de  simula\c{c}\~oes dos modelos que tratamos em sala de aula.
\\
A simula\c{c}\~ao mostra se o modelo que estamos criando faz sentido f\'isico ou n\~ao. Apenas a visualiza\c{c}\~ao das equa\c{c}\~oes que comp\~oem o modelo nem sempre s\~ao t\~ao claras ou suficiente para uma valida\c{c}\~ao das hip\'oteses que consideramos.
\\
Al\'em disso, a simula\c{c}\~ao \'e a parte mais divertida e gratificante da resolu\c{c}\~ao dos problemas que nos deparamos nessa disciplina.
%-----------------------------------------------------------------
\end{enumerate}
\end{enumerate}
\end{document}
