\documentclass[a4paper]{article}
\usepackage[brazilian]{babel}
\usepackage{subfig}
\usepackage{booktabs}
\usepackage{graphicx}
%-----------------------------------------------------------------
\title{MS 680 - Modelos matem\'{a}ticos aplicados a Biologia}
\author{122830 Alcides Goldoni Junior\\
  \small Primeira prova \\
}%Fechando Title
\begin{document}
\maketitle
%-----------------------------------------------------------------
%Seções
%-----------------------------------------------------------------
\begin{enumerate}

%Questao 1-----------------------------------------------------------------
\item
\\
Para construir um modelo de conv\'ivio de esp\'ecies onde existe a competi\c{c}\~ao entre elas, primeiramente, vamos determinar as tr\^es esp\'ecies envolvidas e as rela\c{c}\~oes entre elas. No nosso caso, temos: gavi\~oes ($G_n$), cobras ($C_n$) e roedores ($R_n$). Gavi\~oes predam as cobras e os roedores; Cobras predam os roedores.\\
Nessa situa\c{c}\~ao, o sistema que modela a tripla din\^amica populacional \'e:
%aqui começa o modelo matematica da dinâmica triplice
\begin{equation}
\left\{\begin{array}{l}
\Delta G_n = \alpha_1 G_n + G_n C_n\beta_2 + G_n R_n\gamma_2 - \frac{\alpha_1 G_n ^2}{K}  - M\\
\\
\Delta C_n = \beta_1 C_n + C_n R_n\gamma_2 - C_n G_n\beta_2 - \frac{\beta_1 C_n ^2}{K} - M \\
\\
\Delta R_n = \gamma_1 R_n - R_n C_n\gamma_2 - G_n R_n\gamma_2 - \frac{\gamma_1 R_n ^2}{K} -M 
\end{array}
\end{equation}
\\
Onde:\\
$G_n$ \'e o n\'umero de gavi\~oes;\\
$C_n$ \'e o n\'umero de cobras;\\
$R_n$ \'e o n\'umero de roedores;\\
$\alpha_1$ \'e a taxa de crescimento dos gavi\~oes;\\
$\alpha_2$ \'e a taxa de decrescimento dos gavi\~oes;\\
$\beta_1$ \'e a taxa de crescimento das cobras;\\
$\beta_2$ \'e a taxa de decrescimento das cobras;\\
$\gamma_1$ \'e a taxa de crescimento das roedores;\\
$\gamma_2$ \'e a taxa de decrescimento das roedores;\\
$K$ \'e a capacidade de suporte do meio;\\
$M$ \'e a quantidade de indiv\'iduos que morrem por causas naturais (velhice, doen\c{c}a, cat\'astrofes, etc.).\\
\\
Os coeficientes $\alpha_1, \beta_1, \gamma_1$ representam a taxa de natalidade da esp\'ecie.\\
O efeito produzido pelo aparecimento de uma outra esp\'ecie, competi\c{c}\~ao inter-espec\'ifica, \'e proporcional ao produto dos termos das esp\'ecies multiplicado pelo fator de decrescimento da esp\'ecia presa. Neste modelo, estou supondo que o decrescimento ($\alpha_2, \beta_2, \gamma_2 $) n\~ao muda mesmo mudando a esp\'ecie predadora.\\
O coeficiente $\frac{P ^2}{K}$, onde $P$ \'e uma popula\c{c}\~ao qualquer, mede o efeito do crescimento de uma popula\c{c}\~ao pelo acr\'escimo de um novo indiv\'iduo.\\
Uma observa\c{c}\~ao relevante para a simula\c{c}\~ao do modelo \'e o valor dado aos coeficientes $\alpha$, $\beta$ e $\gamma$, pois cada esp\'ecie pode ter a capacidade de excluir a outra.\\
No caso de n\~ao existirem predadores, as presas crescem at\'e se estabilizarem devido a capacidade de suporte.\\
A popula\c{c}\~ao dos predadores diminuem na aus\^encia de presas.\\
%Questao 2-----------------------------------------------------------------
\item
\\
Considerando um lago com peixes atrativos para pesca e a exist\^ecia de pescadores, vamos criar um modelo que descreve a intera\c{c}\~ao entre peixe e pescadores levando em considera\c{c}\~ao que na aus\^encia de pescadores os peixes possuem crescimento log\'istico e com a presen\c{c}a dos pescadores, o crescimento dos peixes se reduz a uma taxa proporcional a popula\c{c}\~ao de peixes e pescadores. Vamos levar em conta tamb\'em que pescadores s\~ao atra\'idos para o lago a uma taxa proporcional a quantidade de peixes no lago, por\'em, s\~ao desencorajados a uma taxa proporcional ao n\'umero de pescadores  que j\'a est\~ao pescando. Dessa forma, o sistema de equa\c{c}\~oes que descreve o modelo \'e:
%aqui começa o modelo matematica da dinâmica triplice
\begin{equation}
\left\{\begin{array}{l}
\Delta P_n = \alpha P_n (1 - \frac{ P_n}{K})  - \beta P_n H_n\\
\\
\Delta C_n = \gamma P_n H_n - \varepsilon H_n  \\
\end{array}
\end{equation}
\\
Onde:
\\
$P_n$ \'e a popula\c{c}\~ao de peixes;\\
$H_n$ \'e a popula\c{c}\~ao de pescadores;\\
$K$ \'e a capacidade de suporte do meio, no nosso caso, do lago;\\
$\alpha$ \'e a taxa de crescimento da popula\c{c}\~ao de peixes; \\
$\beta$ \'e a taxa de decrescimento da popula\c{c}\~ao de peixes; \\
$\gamma$ \'e a taxa de decrescimento da popula\c{c}\~ao de pescadores; \\
$\varepsilon$ \'e a taxa com que os pescadores s\~ao desencorajados a pescar. 
\\
\\
O termo $\alpha P_n (1 - \frac{ P_n}{K})$ representa o crecimento log\'istico da popula\c{c}\~ao de peixes, caso n\~ao exista pescadores. O efeito produzido pelo aparecimento dos pescadores \'e proporcional ao produto das popula\c{c}\~oes (peixe e pescadores) multiplicado por um fator de descrecimento da popula\c{c}\~ao de peixes.
\\
A taxa com que os pescadores s\~ao atra\'idos para o lago \'e representado pelo produto das popula\c{c}\~oes (peixe e pescadores) multiplicado por um fator de crescimento dos pescadores. J\' a a diminui\c{c}\~ao dos pescadores \'e proporcional a taxa de pescadores j\'a existe multiplicado por um fator de desencorajamento.
\\
%Questao 3-----------------------------------------------------------------
\item
\\
O que achei mais \'util nessa disciplina \'e que ela \'e a \'unica que estou fazendo nesse semestre que tem, de fato, uma aplica\c{c}\~ao no mundo real, deixando de tratar problemas bobos em exerc\'icios propostos no livro texto e tratando problemas que podem realmente acontecer no mundo real. Essa abordagem de problemas reais \'e o que acho mais importante, n\~ao s\'o nessa disciplina, mas tamb\'em em qualquer outra da matem\'atica aplicada e muitas vezes n\~ao \'e o que acontece.
\\
J\'a o que achei menos importante foi
%Questao 4-----------------------------------------------------------------
\item
\\
Minha maior dificuldade em rela\c{c}\~ao ao primeiro projeto foi pensar em quais hip\'oteses utilizar e como inserir essas hip\'oteses, que s\~ao fen\^omenos reais, nas equa\c{c}\~oes, dessa forma, achei melhor fazer o projeto com as hip\'oteses que foram utlizadas em sala de aula, pois j\'a estava familiarizado com a "cara" que a equa\c{c}\~ao teria.
%Questao 5-----------------------------------------------------------------
\item
\\
Por se tratar de uma disciplina que trabalha com problemas reais, o ponto que vejo como o mais fraco \'e a falta de  simula\c{c}\~oes dos modelos que tratamos em sala de aula.
\\
A simula\c{c}\~ao mostra se o modelo que estamos criando faz sentido f\'isico ou n\~ao. Apenas a visualiza\c{c}\~ao das equa\c{c}\~oes que comp\~oem o modelo nem sempre s\~ao t\~ao claras ou suficiente para uma valida\c{c}\~ao das hip\'oteses que consideramos.
\\
Al\'em disso, a simula\c{c}\~ao \'e a parte mais divertida e gratificante da resolu\c{c}\~ao dos problemas que nos deparamos nessa disciplina.
%-----------------------------------------------------------------
\end{enumerate}
\end{enumerate}
\end{document}
